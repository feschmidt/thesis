\section{DNA-like oxidation of MoRe thin films}
\label{sec:more}

During his times as postdoc at TU Delft, my supervisor Gary Steele found that the superconducting alloy molybdenum-rhenium was the only suitable superconductor to sustain the growth conditions of carbon nanotubes.
Moreover, the work function of MoRe matches that of graphene quite close, which was why its use was further pushed in the field of hybrid carbon-superconductor devices.
However, it was often overlooked that devices made from MoRe began exhibiting peculiar "spots" visible under an optical microscope after a certain while.


\begin{figure}
	\centering
	\includegraphics[]{{chapter-experimental-methods/figs-fabrication/placeholder.svg}.png}
	\caption{
		\textbf{DNA-like oxidation of MoRe thin films.}
		\textbf{A,} Optical image of MoRe film under bright (left) and dark field (right).
		\textbf{B,} High-magnification image of MoRe film.
		\textbf{C,} AFM image of one of the crystals.
		\textbf{D,} EDX of a MoRe oxidation crystal, revealing that the oxidization occurs from the rhenium part, while the molybdenum stays inert.
	}
	\label{fig:placeholder}
\end{figure}


TODO: Short text of my old presentation, together with EDX of Miguel

We therefore chose to abandon using MoRe for our devices, and switched to NbTiN or aluminum, depending on the specific circuit requirements.