\section{Measurement setup}

\dropcap{O}{nce} the experimentalist has deemed their sample worthy of measuring, there are a few battles to be fought outside the cleanroom and in the measurement laboratory.

\subsection{Electronic noise}


\begin{figure}
	\centering
	\includegraphics[]{{chapter-experimental-methods/figs-fabrication/placeholder.svg}.png}
	\caption{
		\textbf{Electronic noise reduction using low-pass filters.}
		\textbf{A,} Measured transfer function of individual homemade copper powder and two-stage RC filters.
		The cut-off frequency for the RC filters is approximately \SI{30}{kHz}, while high-frequency noise leaks through above \SI{1}{MHz}.
		The copper powder filters significantly suppress frequencies above \SI{1}{GHz} and above.
		\textbf{B,} Photograph of a two-stage RC filter of SMD 0812 elements.
		\textbf{C,} Photograph of the individual components of a copper powder filter, and a fully assembled one.
	}
	\label{fig:placeholder}
\end{figure}

\subsection{Fridge wiring}

\begin{figure}
	\centering
	\includegraphics[]{{chapter-experimental-methods/figs-fabrication/placeholder.svg}.png}
	\caption{
		\textbf{Cryogenic fridge wiring.}
		\textbf{A,} Inside view of a Bluefors LD dry dilution refridgerator.
		\textbf{B,} Photograph of a four-port PCB mounted on a copper block with rails.
		\textbf{C,} Fully enclosed sample mounted inside of a hand-wound magnet and bolted to the millikelvin stage of the dilution refridgerator.
	}
	\label{fig:placeholder}
\end{figure}

All measurements presented in this thesis were performed in dry dilution refridgerators from \textit{Oxford Instruments} or \textit{Bluefors Oy} at millikelvin temperatures.
