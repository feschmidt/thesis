\section{Device fabrication}
\label{sec:fabrication}

\subsection{The art of making encapsulated graphene devices}

\begin{figure}
	\centering
	\includegraphics[]{{chapter-experimental-methods/figs-fabrication/placeholder-large.svg}.png}
	\caption{
		\textbf{From bulk crystals to van der Waals heterostructures.}
		\textbf{A,} Bulk crystals of HOPG (grey) and hBN (white), together with a piece of wafer adhesive tape used to thin down these crystals by repeated folding and opening of the tape.
		The tape is pressed on a $6\times 6\si{mm\squared}$ piece of silicon with \SI{300}{nm} \ce{SiO2} to enhance the optical contrast for monolayer graphene.
		\textbf{B,} Optical microscope image of a thin flake of graphite, exhibiting regions of single and multilayer graphene.
		\textbf{C,} Optical microscope image of an hBN flake of approximately \SI{30}{nm} thickness.
		All scalebars \SI{20}{\micro m}.
		\textbf{D,} Photograph of a glass slide with PDMS covered with a spun-on layer of PPC (approximately 1-2\si{\micro m} thick).
		\textbf{E,} Typical sandwich assembly cycle for creating multi-layered van der Waals heterostructures:
		\textbf{i,} Polymer brought in contact with exfoliated flake on substrate at room temperature.
		\textbf{ii,} Heating substrate above glass transition temperature $T_g$ enhances the adhesion of the flake to the polymer significantly above the adhesion to the substrate.
		\textbf{iii,} Subsequent cooling of the substrate leads to stiffening of the polymer.
		Combined with rapid lifting of the glass slide, usually solely induced by the thermally shrinking polymer, lifts the flake off the substrate.
		By repeating steps \textbf{i}-\textbf{iii}, multiple flakes can be stacked on top of each other.
		\textbf{iv,} In order to deposit the finished heterostructure on a final substrate, the heterostructure is brought in contact with the substrate at room temperature, and the stage heated above the melting temperature of the polymer.
		Slow lifting of the glass slide leads to the structure remaining on the substrate, while the polymer can either remain fully stuck to the PDMS, or also remain on the substrate.
		\textbf{v,} Polymer residues can be removed in organic solvents such as anisole, NMP, PRS3000 or chloroform.
	}
	\label{fig:placeholder-large}
\end{figure}


\begin{figure}
	\centering
	\includegraphics{{chapter-experimental-methods/figs-fabrication/placeholder-large.svg}.png}
	\caption{
		\textbf{Fabrication issues associated with van der Waals assembly.}
		\textbf{A,} Due to the good adhesion of the polymer to metal surfaces, after flake deposition the polymer often gets delaminated and the chip needs to be thoroughly cleaned using wet chemistry solvents.
		\textbf{B,} Due to the low adhesion of BN to sapphire and silicon nitride, many stacks were washed off the final substrates during solving the polymer.
		We did not observe this behaviour on silicon substrates, but had a yield of less than 50\% on sapphire and 0\% on silicon nitride films.
		\textbf{C,} Atomic force microscope image of a BN-G-BN sandwich assembled via the dry PPC transfer method.
		Many small bubbles are visible
		\textbf{D,} AFM image of a heterostructure assembled via the PC method.
		Due to the significantly slower expansion of the polymer, gas and water molecules at the interfaces between the flakes are pushed towards the edges and fewer, but larger pockets are formed.
		\textbf{E,} CAD overlay on the AFM image to determine the position of the metal layers with respect to the flake position.
	}
	\label{fig:placeholder-large}
\end{figure}

Figure 1: Exfoliation of single flakes and sandwich assembly using the dry transfer method.
In the main text, also describe the PC+Chloroform method.
Huge hurdles during this project: Bad flake adhesion on sapphire, very good adhesion between metal ground planes and PPC.

Figure 2: (a) AFM and optical images of sandwiches, illustrating bubbles. (b) overlay with CAD (c) cross-section of final devices.

\begin{figure}
	\centering
	\includegraphics[]{{chapter-experimental-methods/figs-fabrication/placeholder-six.svg}.png}
	\caption{
		\textbf{From stack to device: Fabrication of van der Waals devices.}
		\textbf{A,} BN-G-BN sandwich on sapphire substrate. The optical micrograph is loaded into a CAD program and aligned with respect to the prepatterned markers.
		\textbf{B,} After electron beam exposure and development, the areas to be metallized are open, while the rest of the substrate remains covered by the resist.
		\textbf{C,} In order to make galvanic contacts to the graphene layer, the chip is placed in a \ce{CHF3}+\ce{O2} plasma, which dry-etches the BN layer.
		Careful etch-rate calibration is required to not over- or under-etch.
		\textbf{D,} Sample after metallization and lift-off.
		\textbf{E,} Bilayer HSQ covering the stack and metal leads as an insulating gate dielectric.
		\textbf{F,} Finalized sample with superconducting gate electrode extending over the entire Josephson junction.
	}
	\label{fig:placeholder}
\end{figure}


Figure 3: Step-by step optical images of exposure, dry-etching, liftoff, shaping, top gate dielectric, top gate metals

Figure 4, if possible: Cross-sectional TEM of one of our devices (courtesy Sonia's lab)

\subsection{Fabrication of current bias cavities}

\begin{figure}
	\centering
	\includegraphics{{chapter-experimental-methods/figs-fabrication/placeholder-six.svg}.png}
	\caption{
		\textbf{Fabrication of shunt-bias microwave cavities.}
		\textbf{A,} MoRe base layer after patterning via \ce{SF6}+\ce{He} dry-etching.
		\textbf{B,} Shunt capacitor dielectric covering parts of the superconducting base layer, after patterning via \ce{CHF3}+\ce{O2} dry-etching.
		\textbf{C,} Finished cavity after lift-off deposition of top-plate of the shunt capacitor.
		\textbf{D,} Full view of an aluminum-based bias cavity shorted to ground on the far end via a superconducting constriction junction.
		All scale bars \SI{50}{\micro\meter}.
		\textbf{E,} SEM image of an aluminum constriction junction.
		Scale bar \SI{100}{\nano\meter}.
	}
	\label{fig:placeholder}
\end{figure}

Figure 1: Step by step optical images of MoRe cavities. (a) base layer after etching, (b) dielectric layer finished, (c) top metal finished.

Figure 2: Al cavities, including Al-ScS fab.


\begin{figure}
	\centering
	\includegraphics{{chapter-experimental-methods/figs-fabrication/placeholder.svg}.png}
	\caption{
		\textbf{Device packaging for electrical measurements.}
		\textbf{A,} Photograph of a wire-bonded chip mounted in a copper sample holder.
	}
	\label{fig:placeholder}
\end{figure}

Figure 3: Wirebonding, packaging