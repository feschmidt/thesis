%\documentclass{scrartcl}
\documentclass{dissertation-edit}
\usepackage{geometry}
\geometry{papersize={174mm,244mm}}
\usepackage{blindtext}
\pagestyle{empty}
\pagecolor{mDarkTeal}
\color{mLightBrown}
\begin{document}

%{\Large Back}
\vspace*{1cm}

%\bfseries
\Large

\noindent This thesis investigates fundamental properties of Josephson junctions embedded in microwave circuits, and contains an application arising from this hybrid approach.
%
Motivated by potential high-frequency applications of Josephson field effect transistors, we used the versatility of superconducting coplanar DC bias cavities to extract previously inaccessible information on phase coherent and subgap mechanisms of graphene Josephson junctions.

\noindent \newline
In addition, we use hybrid bias cavity -- Josephson junction devices to detect small, low-frequency currents with aluminum constriction Josephson junctions.
%
Our current sensor is competitive with state-of-the-art techniques and lays the ground work for orders of magnitude better values in future devices.

\noindent \newline
With detailed information on fabrication, material properties and measurement setup, we supply thorough information on the entire process from design to data analysis.

\vspace{2.5cm}\noindent\newline\centering
Casimir PhD Series 2020-15\\
ISBN 978-90-8593-442-4\\
\includegraphics[height=2cm]{TU_P5_white}

\end{document}