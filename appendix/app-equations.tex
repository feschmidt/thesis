\chapter{Mathematical derivations}
\label{app:eqs}
\clearpage
\section{Andreev bound states}

\dropcap{W}{e} here derive equation \ref{eq:ABS-stop} by starting from eq. \ref{eq:ABS-start}. We have

\begin{align}
0 = &-4(\tau-2)\cos(\delta) +\tau(3+\cos(2\delta)) \\
= &(8-4\tau)\cos(\delta)+3\tau+\tau(2\cos^2(\delta)-1) \\
= &\cos^2(\delta)(2\tau)+\cos(\delta)(8-4\tau)+2\tau
\end{align}

Let $\cos(\delta)=x$:

\begin{eqnarray}
0=x^2(2\tau)+x(8-4\tau)+2\tau \\
\Rightarrow x_{\pm}=\frac{\tau-2\pm2\sqrt{1-\tau}}{\tau} \in [-1,1]
\end{eqnarray}
and $\tau\in [0,1]$. We can readily convince ourselves, that the $x_{-}$ solution results in $x<0$ (e.g. for $\tau=0.5$), hence the only possible solution is $x_{+}$. Therefore\footnote{Here we made use of $\tau-2+2\sqrt{\tau-1}=-(2-\tau-2\sqrt{1-\tau})=-((1-\tau)-2\sqrt{1-\tau}+1)=-(\sqrt{1-\tau}-1)^2$.}

\begin{eqnarray}
\delta=\arccos\left[ - \frac{(\sqrt{1-\tau}-1)^2}{\tau} \right].
\label{eq:app:delta}
\end{eqnarray}

We can now reinsert eq.\ref{eq:app:delta} into \ref{eq:CPR-full} to find the maximum possible current value as a function of the transmission:\footnote{Here we use $\cos(\arccos(x))=x$, $\sin^2( \frac{x}{2} )=\frac{1}{2}(1-\cos(x))$ and $\sin(\arccos(x))=\sqrt{1-x^2}$.}

\begin{eqnarray}
\frac{I}{\Delta/(4\Phi_0)} = \frac{\tau\sin(\delta(\tau))}{\sqrt{1-\tau\sin^2(\delta(\tau)/2)}} = \frac{\tau\sqrt{1-x_{+}^2}}{\sqrt{1-\tau/2(1-x_{+})}}
\end{eqnarray}

From now on, it's a matter of plugging in variables and avoiding calculation mistakes by minimizing the need to expand all brackets:

\begin{eqnarray}
\frac{\tau\sqrt{1-x_{+}^2}}{\sqrt{1-\tau/2(1-x_{+})}} \\
= \frac{ \tau\sqrt{1-(\sqrt{1-\tau}-1)^4/\tau^2} }{ \sqrt{1-\tau/2+\tau/2(-(\sqrt{1-\tau}-1)^2/\tau)} } \\
	= \sqrt{ \frac{ \tau^2-(\sqrt{1-\tau}-1)^4 }{ 1-\frac{1}{2} \left( \tau+(\sqrt{1-\tau}-1)^2 \right) } } \\
	= \sqrt{ \frac{ \left\lbrack \tau+(\sqrt{1-\tau}-1)^2 \right\rbrack \left\lbrack \tau-(\sqrt{1-\tau}-1)^2 \right\rbrack }{ 1-\frac{1}{2} \left( \tau+(\sqrt{1-\tau}-1)^2 \right) } } \\
	= \sqrt{\frac{(a+b)(a-b)}{1-\frac{1}{2}(a+b)}} \\ 
	= \sqrt{\frac{2(a+b)(a-b)}{2-(a+b)}},
\end{eqnarray}

where we defined\footnote{There is no deeper reason for doing this, other than avoiding to explicitly calculate the fourth order bracket.}

\begin{eqnarray}
a =& \tau \\
b =& (\sqrt{1-\tau}-1)^2 \\
a+b =& 2(1-\sqrt{1-\tau}) \\
a-b =& 2(\tau-1+\sqrt{1-\tau}).
\end{eqnarray}

Plugging this back in, we arrive at

\begin{eqnarray}
\frac{4\Phi_0I}{\Delta} =& \sqrt{ \frac{ 2\cdot2(1-\sqrt{1-\tau}) \cdot 2(\tau-1+\sqrt{1-\tau}) }{ 2-2(1-\sqrt{1-\tau}) } } \\
\frac{1}{4}\left( \frac{4\Phi_0I}{\Delta} \right)^2 =& \frac{ (1-\sqrt{1-\tau})(\tau-1+\sqrt{1-\tau}) }{ 1-1+\sqrt{1-\tau} } \\
=& \frac{\tau-1+\sqrt{1-\tau}-\tau\sqrt{1-\tau}+\sqrt{1-\tau}-(1-\tau)}{\sqrt{1-\tau}} \\
=& \frac{2(1-\tau)+(1-\tau)\sqrt{1-\tau}+\sqrt{1-\tau}}{\sqrt{1-\tau}} \\
=& (1-\tau)-2\sqrt{1-\tau}+1 \\
=& (\sqrt{1-\tau}-1)^2.
\end{eqnarray}

Neglecting the solution $I_c<0$, we finally arrive at the expression for the maximum current supported by a single Andreev mode of transmission $\tau$, eq.\ref{eq:ABS-stop}:

\begin{equation}
I_c=\frac{\Delta}{2\Phi_0} \left( 1-\sqrt{1-\tau} \right).
\end{equation}

\references{dissertation}

