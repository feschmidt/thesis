\newchapstyle
\chapter{Current phase relations of graphene Josephson junctions in microwave circuits}
\label{chap:gJJ-CPR}

\blfootnote{
	\color{title}
	This chapter is based on previously unpublished data of the devices presented in Chapter~\ref{chap:gJJ}.
	%
	Data and code to reproduce the calculations and figures presented here can be found on Zenodo~\cite{schmidtDataCodeCurrent2020}.
}

\begin{abstract}
	We perform extensive analysis of graphene Josephson junctions embedded in microwave circuits.
	%
	By comparing a diffusive junction at \SI{15}{\milli\kelvin} with a ballistic one at \SI{15}{\milli\kelvin} and \SI{1}{\kelvin}, we are able to reconstruct the current-phase relation.
\end{abstract}

%% Start the actual chapter on a new page.
\newpage

\section{Introduction}

Josephson junctions (JJs) are widely used in radio frequency (RF) applications, such as quantum information processing and sensing, where they are being exploited as nonlinear inductors.
%
For the use of JJs in superconducting quantum information circuits, the junction nonlinearity has a major effect on the circuit requirements and capabilities~\cite{kringhojAnharmonicitySuperconductingQubit2018}.
%
However, the nonlinear character can significantly differ from junction to junction depending on the intrinsic properties governed by their current-phase relation (CPR).

Specifically, in systems such as graphene Josephson junctions (gJJs), the CPR can exhibit a forward skew compared to the case of a purely sinusoidal CPR such as in superconducting tunnel junctions like aluminum oxide~\cite{ginzburgDeterminationCurrentPhase2018}.
%
While the CPR of gJJs has been studied in the DC regime~\cite{englishObservationNonsinusoidalCurrentphase2016,nandaCurrentPhaseRelationBallistic2017}, and gJJs have been successfully incorporated in RF circuits~\cite{schmidtBallisticGrapheneSuperconducting2018,krollMagneticFieldCompatible2018,wangCoherentControlHybrid2019}, the influence of the potentially skewed CPR has not been studied in the latter.

Here, we analyze the effect of a nonlinear CPR on the microwave performance of gJJ embedded in microwave circuits.
%
Measuring two devices in different states, we compare the influence of scattering transport and temperature on the JJ nonlinearity.
%
Our circuit design allows in-situ, and even simultaneous, DC and RF measurements, providing us with various measurement types to compare.
%
The results show the usefulness of combining DC and RF in the same circuits for fundamental research on Josephson junction physics, which distinguishes it from pure RF CPR measurements~\cite{rifkinCurrentphaseRelationPhasedependent1976}.

\section{Circuit characterization}

Our circuit consists of a DC-bias microwave cavity formed by a coplanar waveguide (CPW) which is shunted by a large capacitor at the input, and shorted to ground on the far end by a gate-voltage ($V_g$) tunable graphene Josephson junction, cf. Fig.~\ref{fig:figure1}(a) and Refs.~\cite{schmidtBallisticGrapheneSuperconducting2018,schmidtCurrentDetectionUsing2020,bosmanBroadbandArchitectureGalvanically2015c}.
%
The superconducting base layer and shunt capacitor metal layers consist of DC-sputtered molybdenum-rhenium on a sapphire substrate, while the shunt capacitor dielectric layer is PECVD-SiN.
%
The gate voltage lead is fed through a second shunt capacitor of the same geometry as the one at the input in order to suppress RF radiation leaking in through or out of the gate line.
%
The RF wiring of both samples was fabricated on a single \SI{2}{inch} sapphire wafer, after which the wafer was diced into \SI{10x10}{\milli\meter} pieces onto which the individual gJJ were placed.
%
The gJJ consist of boron nitride encapsulated single layer graphene with self-aligned side-contacts of DC-sputtered niobium titanium nitride (NbTiN), fabricated via the etch-fill technique\cite{wangOneDimensionalElectricalContact2013b,schmidtBallisticGrapheneSuperconducting2018}
%
The gJJ are designed to be \SI{5}{\micro\meter} wide and separate the NbTiN leads by a length of \SI{500}{\nano\meter}.
%
Gate tunability is achieved by placing a third NbTiN lead extending over the entire gJJ, separated by a bilayer of HSQ.
%
The circuit is mounted on the millikelvin plate on a dilution refrigerator and connected to the outside world via a bias-T, allowing both DC and RF characterization in the same setup.

We measured two separate devices with nominally identical microwave circuits and junction designs:
%
One of the devices exhibited signatures of ballistic transport in form of Fabry-Pérot-like oscillations, which we will refer to as the \textit{ballistic device}, which is the device presented in the main text of Ref.~\cite{schmidtBallisticGrapheneSuperconducting2018}.
%
The other one, in lack of such features, will be called \textit{diffusive device}, and corresponds to the reference sample of the same reference, cf. Supplementary Material Sec.~\ref{sec:ballistic} and Fig.~\ref{fig:SMFig-ballistic} for details on the ballistic features.

We extract the DC circuit parameters by applying a bias current to the JJ, using the CPW as a long capacitive lead and measuring the voltage drop across the gJJ.
%
All DC lines were equipped with $\pi$-filters in the room temperature battery powered electronics, as well as copper powder and two-stage RC filters thermally anchored to the millikelvin stage of the dilution refrigerator.
%
When exceeding a critical current, the JJ switches from the zero-voltage to the resistive state.
%
We record this switching current $I_s^{\rm DC}$ for varying gate voltages, as depicted in Fig.~\ref{fig:figure1}(b,c) for the two devices at a base temperature of \SI{15}{\milli\kelvin} in the case of the diffusive, and both base temperature and \SI{1}{\kelvin} for the ballistic device.
%
The DC switching current of the diffusive device ranges from a few \SI{100}{\nano\ampere} to \SI{5.5}{\micro\ampere}, similar to the ballistic device at \SI{1}{\kelvin}.
%
At base temperature, the maximum $I_s^{\rm DC}$ of the ballistic device reaches up to \SI{7.5}{\micro\ampere}.
%
Both samples exhibit significantly larger switching current for $V_g>V_{\rm CNP}$ (n-doping) compared to $V_g<V_{\rm CNP}$ (p-doping), where $V_{\rm CNP}$ denotes the gate voltage at the charge neutrality point (CNP) of the gJJ.
%
We attribute this to a reduced contact transparency in the p-doped regime~\cite{schmidtBallisticGrapheneSuperconducting2018}.
%
We measure $V_{\rm CNP}^{\rm diff}=\SI{1.55}{\volt}$ and $V_{\rm CNP}^{\rm ball}=\SI{-1.39}{\volt}$ for the diffusive and ballistic sample, respectively.
%
Discrepancies are presumably due to differences in residual doping during fabrication.

\begin{figure}[t]
	\centering
	\includegraphics[width=\linewidth]{chapter-gJJ-CPR/figs/Figure1}
	\caption{
		\textbf{A graphene Josephson junction embedded in a DC bias microwave circuit.}
		%
		\textbf{(a)} Measurement schematic.
		%
		The gJJ shorts a coplanar waveguide transmission line to ground, which forms a gate-tunable $\lambda/2$-resonator.
		%
		$V_g$ is fed through an additional shunt capacitor (not shown).
		\textbf{(b,c)} Switching current for the diffusive \textbf{(b)} and ballistic Josephson junction \textbf{(c)}, at base-temperature of \SI{15}{\milli\kelvin} (blue) and at \SI{1}{\kelvin} (red).
		%
		\textbf{(d,e)} Resonance frequencies versus gate voltage for the diffusive \textbf{(d)} and ballistic \textbf{(e)} device.
		%
		The gate-tunable Josephson inductance changes the boundary condition of the $\lambda/2$-resonator, thus changing the resonance frequency of the circuit.
		%
		Dashed grey lines indicate the charge neutrality point of each device.
	}
	\label{fig:figure1}
\end{figure}

For high frequency signals, i.e. a few \si{\giga\hertz}, the gJJ behaves as a nonlinear inductor, with Josephson inductance
%
\begin{align}
L_J = \frac{\Phi_0}{2\pi}\left(\diff{I_J}{\delta}\right)^{-1},
\label{eq:LJgeneral}
\end{align}
%
where $\Phi_0$ is the magnetic flux quantum and $I_J(\delta)$ the current phase relation (CPR) between the phase difference $\delta$ across the JJ between two superconducting banks and the corresponding supercurrent $I_J$ flowing through the junction.
%
Depending on the impedance of the gJJ at the circuit resonance frequency, $Z_J=i\omega_0 L_J$, the fundamental mode hosted by the gJJ-terminated CPW varies between a $\lambda/2$ wave for $Z_J\rightarrow0$ and $\lambda/4$ for $Z_J\rightarrow\infty$.

The resonance frequency of a $\lambda/2$-resonator shorted to ground by a Josephson inductance can be approximated by
\begin{align}
f_0\left(I_b,I_c\right) = f_{\lambda/2} \frac{L_r+L_J\left(I_b, I_c\right)}{L_r +  2L_J\left(I_b, I_c\right)}
\label{eq:Pogorzalek}
\end{align}
%
with $L_r$ the bare CPW inductance and $f_{\lambda/2}$ the resonance frequency of the CPW without the JJ, see Supplementary Material Sec.~\ref{sec:validity}.
%
$I_b$ is the bias current flowing through CPW and the JJ, $I_c$ the critical current of the JJ.
%
We can immediately see that for small $L_J$, $f_0\rightarrow f_{\lambda/2}$, while for $L_J\gg L_r$, $f_0 \rightarrow f_{\lambda/2}/2$.

The circuit response is measured by recording the reflection coefficient $S_{11}$ of the cavity using a vector network analyzer, which excites the device through a series of attenuators and a directional coupler, and measures the reflected signal, amplified by low noise cryogenic and room temperature HEMTs. 
%
We fit the response using an analytical model to extract resonance frequency $f_0$ and internal ($\kappa_i$) and external loss rates($\kappa_e$) (cf. Supplementary Material Sec.~\ref{sec:extraction}).
%
We observe gate-tunable resonance frequency $f_0$ between \SIrange{7.0}{8.2}{\giga\hertz}, comparable for both devices, cf. Fig.~\ref{fig:figure1}(d,e).
%
Due to the inverse nature of junction current and inductance, the large changes in $I_s^{\rm DC}$ for $V_g>V_{\rm CNP}$ only lead to minor changes in $f_0$ when comparing the hot and cold ballistic device.
%
On the other hand, even small changes in the significantly smaller $I_s^{\rm DC}$ for $V_g<V_{\rm CNP}$ significantly reduce $f_0$ in this regime.

\section{Measuring the Josephson inductance in DC and RF}

\begin{figure}[t]
	\centering
	\includegraphics[width=\linewidth]{chapter-gJJ-CPR/figs/Figure2}
	\caption{
		\textbf{Resonance frequency vs switching currents for two different gJJ devices.}
		%
		Both the diffusive device at low temperature \textbf{(a)} and the ballistic device at \SI{1}{\kelvin} (\textbf{(b)}, red) show monotonically increasing $f_0$ versus DC-extracted switching currents.
		%
		In contrast, for low temperatures, the ballistic gJJ (\textbf{(b)}, blue) exhibits multi-valued $f_0\left(I_s\right)$ for gate voltages larger (full circles) and smaller (empty squares) than the charge neutrality point.
		%
		The	multivalued behavior in the ballistic device at low temperature presumably originates from significant differences in junction transparency between n- and p-doping, and only allows for a fit for $V_g>0$.
		%
		This is not observed at higher temperature or for the diffusive device.
		%
		Dashed lines correspond to fits to Eq.~\ref{eq:Pogorzalek}.
	}
	\label{fig:figure2}
\end{figure}

Assuming a purely sinusoidal current-phase relation,
%
\begin{align}
I_J(\delta) = I_c\sin\delta,
\label{eq:CPR-sin}
\end{align}
%
the Josephson inductance can be extracted from the current phase relation via
%
\begin{align}
L_J = \frac{\Phi_0}{2\pi I_c \cos\delta}.
\label{eq:LJsin}
\end{align}
%
However, depending on the exact shape of the CPR, $L_J$, and with it $f_0$, can significantly deviate from the above equations, cf. Supplementary Fig.~\ref{fig:SMinfluence}.
%
For example, the CPR of gJJ are known to exhibit forward-skewing, where the skew is defined as the deviation of the CPR maximum from phase $\pi/2$, $S=2\delta_{\rm max}/\pi -1$.
%
This leads to a reduced slope of the CPR around zero phase, which enhances $L_J$ compared to the case of a sinusoidal CPR for the same value of $I_c$.

\begin{figure}[t]
	\centering
	\includegraphics[width=\linewidth]{chapter-gJJ-CPR/figs/Figure3}
	\caption{
		\textbf{Comparing DC-switching and RF-critical currents.}
		%
		RF-extracted critical current versus DC-measured switching current for the diffusive device at \SI{15}{\milli\kelvin} \textbf{(a)} and the ballistic device at \SI{15}{\milli\kelvin} and at \SI{1}{\kelvin} (\textbf{(b)}, blue and red, respectively).
		%
		Full circles (empty squares) correspond to $V_g>V_{\rm CNP}$ ($V_g<V_{\rm CNP}$).
		%
		Dashed line corresponds an equivalency between $I_s^{\rm DC}$ and $I_c^{\rm RF}$.
		%
		Both scattering in the JJ as well as elevated temperatures reduce the deviation from RF and DC data.
	}
	\label{fig:figure3}
\end{figure}

As depicted in Fig.\ref{fig:figure2}, we can fit the RF-measured $f_0$ versus the DC-measured $I_s^{\rm DC}$ using Eqs.~\ref{eq:Pogorzalek}-\ref{eq:LJsin} assuming a sinusoidal CPR, shown by the black dashed line.
%
Here, we assumed $\delta=0$ because the DC bias port was shorted to ground during measurement, and the RF excitation merely oscillates the phase around zero.
%
While the extracted values for $f_{\lambda/2}$ for the three cases only vary by \SI{1}{\percent}, we measure a parameter spread of up to \SI{51}{\percent} for $L_r$, see Supplementary Material Tab.~\ref{tab:frLr}.
%
We must therefore conclude that the assumed values for $L_J(I_s^{\rm DC})$ are different for the three devices and do not originate from the same, sinusoidal, CPR.
%
We also note that the fit only converges for the ballistic device at \SI{1}{\kelvin} and the diffusive, whereas, at \SI{15}{\milli\kelvin}, the ballistic device exhibits multi-valued $f_0\left(I_s\right)$, as the resonance frequency and switching current follow different trends for p- and n-doping.

Additionally, we can directly compare the estimates for the junction critical current from our DC measurement, $I_s^{\rm DC}$, with the value calculated from our RF measurement of the resonance frequency via Eq.~\ref{eq:LJsin}, $I_c^{\rm RF}=I_c\left(L_J\left(f_0\right)\right)$. 
%
As shown in Fig.~\ref{fig:figure3}, our devices exhibit tunable supercurrent over almost two orders of magnitude range.
%
In the diffusive device, $I_s^{\rm DC}$ and $I_c^{\rm RF}$ match closely, but small deviations at the low and high end are visible, with $I_c^{\rm RF}$ resulting in values larger than expected from DC.
%
Similarly, the DC measurements of the ballistic device at \SI{1}{\kelvin} underestimate the critical current as extracted from RF for small values of $I_s^{\rm DC}$, but match remarkably well everywhere else.
%
This matches with the expectation of reduced forward skewing of the CPR at higher temperatures:
%
The skew is due to the phase coherence of Andreev bound states traversing the normal region between the superconducting banks multiple times (or, in a similar picture, multiple ABS crossing the normal region) which in turn means a longer phase coherence length is required to keep this contribution.
%
As the phase coherence length is highly sensitive to temperature, an increase in the latter results in both a reduction of switching current and forward skewing~\cite{fuechsleEffectMicrowavesCurrentPhase2009,hagymasiJosephsonCurrentBallistic2010,black-schafferStronglyAnharmonicCurrentphase2010,rakytaMagneticFieldOscillations2016,englishObservationNonsinusoidalCurrentphase2016}.
%
At base temperature, the ballistic device deviates significantly from the DC-calculations, especially at low currents and $V_g<0$.
%
Deviations at large $I_c$ are less obvious, presumably due to the fact that in this regime, $L_J \ll L_r$ and the Josephson inductance has only minor effect on $f_0$.

At first glance, any forward skew in the CPR should lead to a drop in CPR slope.
%
Therefore, for the same value of $I_c$, $L_J$ should increase and $I_s^{\rm RF}<I_s^{\rm DC}$.
%
However, our fit results in the opposite behavior, cf. Supplementary Fig.~\ref{fig:SMtau}:
%
The fit model returns a larger $I_c$ than expected for a sinusoidal CPR, given that the data to be fitted has an underlying forward-skewed CPR.
%
This is achieved by returning incorrect values for $f_{\lambda/2}$ and $L_r$ which explains to mismatching values from Fig.~\ref{fig:figure2}.
%
In lack of a reference device with a simple short to ground instead of a gJJ, or a JJ with known sinusoidal CPR, we need additional measurements to accurately determine the CPRs of our devices.
%
This shows that extrapolation of a value for $L_J$ at microwave frequencies from DC measurements can lead to severe discrepancies.
%
In order to examine the underlying mechanisms further, we continue by studying the power and bias current dependence of our circuit.



\section{Reconstructing the current phase relation}

\subsection{Anharmonicity of a gJJ RF circuit}

The nonlinear inductance of a Josephson junction consequently introduces nonlinear behavior of the overall circuit.
%
Depending on the exact circuit design and participation ratio between Josephson and total circuit inductance, this nonlinearity is more or less diluted, yet finite so-called anharmonicity $\beta$, i.e. deviation from the ideal case of pure LC-resonator behavior, remains.
%
Our circuit architecture allows us to extract this quantity directly and to calculate the expected CPR skew.

We can observe the anharmonicity of our DC bias circuit terminated with the diffusive gJJ by performing $S_{11}$ measurements at high drive powers for a series of different gate voltages, as shown in Fig.~\ref{fig:figure4} for $V_g=\SI{+10}{\volt}$.
%
At very low drive powers, $\beta$ has negligible effect on the circuit response, which can still be described by a purely harmonic oscillator here.
%
With increasing on-chip power $P_{\rm in}$, the resonance frequency experiences a down-shift, and both amplitude and phase of $S_{11}$ start to get skewed towards lower frequencies.
%
Once $P_{\rm in}$ exceeds a critical threshold, the resonator response bifurcates, which can be seen by the discontinuity in the data.
%
For reference, all other measurements of this device were performed at $P_{\rm in}\approx\SI{-131.4}{dBm}$, still in the linear regime and with a maximum current at the junction of $I_{\rm RF}\approx\SI{3.0}{\nano\ampere}$ well below the critical current, cf. Supplementary Material Fig.~\ref{fig:SMFigpoweratJJ}.

Using the previously determined parameters $f_0$, $\kappa_i$ and $\kappa_e$, we can model the data by solving the equation of motion of a harmonic oscillator with an additional third order term in the cavity field with amplitude $\beta$, 
%
\begin{align}
\dot{\alpha} = \left[ -i \left( \Delta+\beta\abs{\alpha}^2 \right)-\frac{\kappa}{2} \right]\alpha + \sqrt{\kappa_\text{e}} S_\text{in}\ ,
\label{eq:Duffing-EOM}
\end{align}
%
as detailed in Supplementary Material Sec.~\ref{sec:SMduffing}, where $S_\text{in}$ is the field amplitude of the drive.
%
We note that best agreement between data and model is reached when introducing nonlinear dissipation in the form of increasing internal linewidth with drive power, $\kappa_i=\kappa_i(S_{\rm in})$, cf. Supplementary Fig.~\ref{fig:SMpower}.
%
This is in contrast with circuits incorporating standard aluminum oxide JJs, where nonlinear dissipation with increasing power is usually absent~\cite{boakninDispersiveMicrowaveBifurcation2007b}.

\begin{figure}[t]
	\centering
	\includegraphics[width=\linewidth]{chapter-gJJ-CPR/figs/Figure4}
	\caption{
		\textbf{Power dependence of a nonlinear microwave device with diffusive gJJ.}
		%
		\textbf{(a)} Absolute value of the reflection coefficient $S_{11}$ versus frequency for increasing drive power.
		%
		Due to the circuit nonlinearity, the resonator experiences a downshift and bifurcation at elevated drive powers.
		%
		Solid lines indicate linecuts in \textbf{(b)} and \textbf{(c)}.
		%
		\textbf{(b-c)} Absolute value \textbf{(b)} and phase \textbf{(c)} of $S_{11}$ for varying drive power as indicated in \textbf{(a)}.
		%
		Linecuts are each offset by $-0.5$ in y for clarity.
		%
		Black lines are fits.
		%
		\textbf{(d)} Anharmonicity vs gate voltage.
		%
		Green dots: data as extracted from fits as in \textbf{(b-c)}, orange line: model using $I_s^{\rm DC}$, black line: model using $L_J$.
		%
		The overestimation of the anharmonicity of $I_s^{\rm DC}$ for high gate voltages is consistent with and confirms a non-sinusoidal CPR.
	}
	\label{fig:figure4}
\end{figure}

There are several dissipation mechanisms known in superconducting microwave circuits that depend on drive power, such as on-chip heating~\cite{portisPowerinducedSwitchingHTS1991,heinFundamentalLimitsLinear1997,wosikPowerHandlingCapabilities1997}, dielectric losses~\cite{martinisDecoherenceJosephsonQubits2005c,oconnellMicrowaveDielectricLoss2008a,gunnarssonDielectricLossesMultilayer2013,lisenfeldElectricFieldSpectroscopy2019}, or subgap losses~\cite{dassonnevilleDissipationSupercurrentFluctuations2013,ferrierPhasedependentAndreevSpectrum2013,dassonnevilleCoherenceenhancedPhasedependentDissipation2018}.
%
Heating of the circuit itself is unlikely since $f_0$ should tune significantly stronger due to a reduced $I_c$ at elevated temperatures, with potentially significant influence on $f_0$, c.f. Fig.~\ref{fig:figure1}, which we did not observe for any of the gate voltages.
%
Moreover, the power dissipated on-chip is extremely small and very unlikely to cause even local heating.

Losses due to electric dipole moments of two-level systems are also unlikely the source of observation, as these are known to be activated for decreasing drive excitation voltages~\cite{martinisDecoherenceJosephsonQubits2005c,oconnellMicrowaveDielectricLoss2008a,gunnarssonDielectricLossesMultilayer2013}.
%
Moreover, TLS mainly reside in disordered dielectric materials.
%
However, there is only dielectric volume present at the shunt capacitor dielectric and the gJJ (encapsulating BN and HSQ top-gate).
%
Here, the circuit has voltage nodes and voltage fluctuations, which could activate the TLS, are expected to have negligible effect on the circuit performance.

We therefore attribute the source of the observed nonlinear damping to low-lying subgap states within the induced superconducting gap in the gJJ.
%
These subgap states can be due to e.g. intransparent superconductor-normal contacts, or Andreev bound states with large transverse momentum, polluting the bulk superconducting gap and leading to microwave loss~\cite{schmidtBallisticGrapheneSuperconducting2018}.
%
As the drive power increases, these subgap states get populated, resulting in an increase in internal loss rate, cf. Supplementary Fig.~\ref{fig:SMFig-lossrates}.
%
Loss mechanisms in similar SNS systems, with normal metal weak links, have shown similar effects~\cite{fuechsleEffectMicrowavesCurrentPhase2009,dassonnevilleDissipationSupercurrentFluctuations2013}, but they have not been observed before in gJJ.

We compare the measured value of $\beta$ with the one expected from a CPW shorted to ground by a Josephson junction, which is given by
\begin{align}
\beta=-\frac{f_0}{2} \left(\frac{L_J}{L_r+L_J}\right)^3\ ,
\label{eq:anharmonicity}
\end{align}
%
where we have followed the same notation as in the earlier equations~\cite{wilsonPhotonGenerationElectromagnetic2010b,zhouHighgainWeaklyNonlinear2014}.
%
As expected, assuming a sinusoidal CPR and using $I_s^{\rm DC}$ to calculate $L_J$ overestimates the anharmonicity ($\beta_{I_s}$), cf. Fig.~\ref{fig:figure4}(d), while Eq.~\ref{eq:anharmonicity} and the data match better for $I_c^{\rm RF}$ ($\beta_{L_J}$).

A more realistic model for the CPR of our devices than Eq.~\ref{eq:CPR-sin} is the one for a ballistic point contact as a function of transparency $\tau$ and temperature $T$:
%
\begin{align}
I_J(\delta,\tau,T) = \frac{\pi\Delta_0}{2 e R_n} \frac{\sin\delta}{\sqrt{1 - \tau \sin^2\delta / 2}} \tanh\left[\frac{\Delta_0}{k_B T} \sqrt{1 - \tau \sin^2\delta / 2}\right]\ ,
\label{eq:CPR-ball}
\end{align}
%
with the superconducting energy gap at zero temperature $\Delta_0$, the Boltzmann constant $k_B$ and normal state resistance $R_n= R_q/N = h/(Ne^2)\approx \SI{25.812}{\kilo\ohm} / N$~\cite{golubovCurrentphaseRelationJosephson2004a,leeUltimatelyShortBallistic2015}.
%
Here, $R_q$ denotes the quantum Hall resistance and $N$ the number of conducting channels.

The transparency $\tau$ is taken as an average transmission probability of the $N$ channels.
%
With a normal state resistance of or devices ranging between \SIrange{35}{350}{\ohm} (depending on gate voltage), we estimate around 74 to 740 conducting channels.
%
This justifies the use of a single averaged transparency parameter $\tau$.

It has been shown for the case of non-sinusoidal CPR, that the Josephson inductance has reduced nonlinearity, which lowers the circuit anharmonicity, resulting in a correction factor of
\begin{align}
\frac{\beta^\prime}{\beta} =  1-\frac{3}{4}\frac{\sum_i\tau_i^2}{\sum_i\tau_i} \rightarrow 1-\frac{3}{4}\tau 
\label{eq:anh_nonsin}
\end{align}
%
where the limes holds for an average $\tau$ of the ensemble of transmission channels~\cite{kringhojAnharmonicitySuperconductingQubit2018}.
%
We calculate the ratio $\beta_{\rm meas}/\beta_{I_s}=0.57\pm0.03$, resulting in an estimated average channel transmission of $\tau=0.24\pm0.04$ or a corresponding CPR forward skew $S = 0.04\pm0.01$ for the diffusive gJJ device, which slightly deviates from a purely sinusoidal CPR, as expected.
%
Power spectra for the ballistic gJJ were not recorded.


\subsection{Bias current dependence}

A second way of reconstructing the CPR is by means of analyzing the bias current dependence of the high frequency circuit response, as this allows for a direct measure of $L_J(\delta(I_b))))$.
%
We model the bias current dependence of both the ballistic and diffusive device at \SI{15}{\milli\kelvin} using Eqs.~\ref{eq:Pogorzalek} and \ref{eq:LJgeneral} with the assumption of a general CPR according to Eq.~\ref{eq:CPR-ball} and using $\tau$ as a free parameter, as shown in Fig.~\ref{fig:figure5}(a).
%
As the transparency increases, the CPR gets flatter, significantly increasing $L_J$ and thus the cavity tuning, cf. Fig.~\ref{fig:SMinfluence}.
%
For details on the fitting algorithm, see the Supplemental Material Sec.~\ref{sec:fitbiascurrent}.

Our model accurately reproduces the data for all measured gate voltages, cf. Fig.~\ref{fig:figure5}(a), which allows us to extract a CPR-transparency parameter $\tau(V_g)$, as plotted in Fig.~\ref{fig:figure5}
(b).
%
We extract an average channel transmission $\tau_{\rm diff}=0.17\pm0.08$ and $\tau_{\rm ball}=0.88\pm0.03$ for the diffusive and ballistic device, respectively, at base temperature.
%
Measurements at \SI{1}{\kelvin} were not performed.
%
This corresponds to a forward skew of $S_{\rm diff}=0.03_{-0.02}^{+0.01}$ and $S_{\rm ball}=0.34_{-0.05}^{+0.03}$ for the diffusive and ballistic device, respectively.
%
These values agree with the ones extracted from the power dependence within error bars, confirming the skew CPR.
%
We note that they are also comparable to the results obtained from DC-measurements of the CPR~\cite{englishObservationNonsinusoidalCurrentphase2016,nandaCurrentPhaseRelationBallistic2017}.
%
Our result is furthermore supported by the fact that the values for the extracted $f_{\lambda/2}$ and $L_r$ of the two devices here differ by only \SI{0.4}{\percent} and \SI{6.1}{\percent}, respectively, much less than for the gate voltage sweeps.

We plot the thus reconstructed current phase relations of both the ballistic and diffusive gJJ in Fig.~\ref{fig:figure5}(d).
%
While the relatively small forward skew of the diffusive device only deviates slightly from the case of a perfectly sinusoidal CPR, the ballistic device shows significant forward skewing.
%
We would like to point out that even though the skew in the diffusive case is only minor, our RF circuit still sensitive enough to these small changes, making its use competitive with DC-measurements of the CPR by using asymmetric SQUIDs or pickup loops.

\begin{figure}[t]
	\centering
	\includegraphics[width=\linewidth]{chapter-gJJ-CPR/figs/Figure5}
	\caption{
		\textbf{Extracting the current phase relation from current-biasing the gJJ microwave circuit.}
		%
		Fitting the bias current dependence \textbf{(a)}, we can extract the junction transparency \textbf{(b)} and corresponding CPR skew \textbf{(c)} for the diffusive (green) and ballistic (blue) gJJ device versus gate voltage.
		%
		\textbf{(d)} Reconstructed current-phase relation for the diffusive (green) and ballistic (blue) device.
		%
		The large transparency of the ballistic JJ leads to significant forward skewing, while the relatively low transparency of the diffusive JJ only results in minor skewing.
	}
	\label{fig:figure5}
\end{figure}

\section{Conclusion}

We were able to reconstruc the CPR of gJJs by embedding the latter in superconducting microwave circuits.
%
Our results show that scattering of charge carriers, as well as elevated temperature, reduce the CPR skew and with it the circuit anharmonicity via the change in nonlinearity of the JJ itself.
%
Our circuit architecture is an attractive candidate for analyzing the CPR of exotic JJs, such as ferromagnetic or topological ones~\cite{golubovCurrentphaseRelationJosephson2004a,sochnikovNonsinusoidalCurrentPhaseRelationship2015,stoutimoreSecondHarmonicCurrentPhaseRelation2018,assoulineSpinOrbitInducedPhaseshift2019,muraniMicrowaveSignatureTopological2019}.
%
Moreover, the influence of high microwave powers on the CPR can be studied straightforward, as this only requires repeating the bias current measurements at various powers.
%
Additionally, the combination of bias current and power dependence should allow to trace out a larger part of the CPR than just around zero phase.
%
Finally, on top of a built-in method of reconstructing phase-sensitive information, incorporating asymmetric gate-tunable SQUIDs instead of single JJs would enable a further in-situ method of probing and reconstructing the CPR.

For the case of graphene Josephson junctions in cQED applications, they seem rather unfeasible for relatively high-power devices such as quantum-limited parametric amplification, as the inherent nonlinear damping due to the subgap states limits the maximum drive power.
%
Future research on improvements towards a hard superconducting gap could advance these applications.
%
Nevertheless, using gJJs in low-power circuits, such as transmon qubits, is still an attractive alternative to the established aluminum oxide JJs, as the gate-tunability and potential sweet spots in the Fabry-Pérot regime could allow for long qubit coherence times.
%
Our results also indicate that scattering mechanisms, such as diffusive transport or higher temperatures, inside of semiconducting JJs in RF circuits could even be beneficial for cQED because the increased circuit anharmonicity could result in more coherent circuits.

%%%%%%%%%%%%%%%%%%%%%%%%%%%%%%%%%%%
% Insert SM.tex contents here

\clearpage
\pagebreak

%\widetext

%\setcounter{equation}{0}
%\setcounter{figure}{0}
%\setcounter{table}{0}
%\setcounter{page}{1}
%\setcounter{section}{0}

%\renewcommand{\thepage}{S\arabic{page}}
%\renewcommand{\thesection}{S\Roman{section}}
%\renewcommand{\thetable}{S\Roman{table}}
%\renewcommand{\thefigure}{S\arabic{figure}}
%\renewcommand{\theequation}{S\arabic{equation}}
%\renewcommand{\bibnumfmt}[1]{[S#1]}
%\renewcommand{\citenumfont}[1]{S#1}

\section{Supplementary Material: Current phase relations of graphene Josephson junctions in microwave circuits}

\subsection{Classification as diffusive or ballistic JJ}\label{sec:ballistic}

As stated in the main text, we define the device as ballistic or diffusive in the presence or absence of Fabry-Pérot-like oscillations.
%
In Fig.~\ref{fig:SMFig-ballistic}, we plot these oscillations after removing a third order background from the data to remove the overall gate-voltage tuning dependence.
%
Both at base temperature and at \SI{1}{\kelvin}, we observe high-frequent, highly correlated oscillations in all off $f_0$, $I_s^{\rm DC}$ and $G_n=R_n^{-1}$ for the \textit{ballistic} device, which justifies its classification as such.
%
For the same voltage range, however, the \textit{diffusive} device only shows a low-frequent trend originating from the deviation about the removed background, thus lacking the ballistic feature.

\begin{figure}[!h]
	\centering
	\includegraphics[width=\linewidth]{chapter-gJJ-CPR/figs/SMFigure-ballistic}
	\caption{
		\textbf{Fabry-Pérot oscillations.}
		%
		\textbf{(a-c)} Oscillations in the resonance frequency, DC-switching current and normal state conductance as a function of gate voltage for the ballistic device at base temperature (blue) and \SI{1}{\kelvin} (red).
		%
		\textbf{(d-f)} For the diffusive device, no such features are observed, only a slowly varying background, justifying the classification as \textit{diffusive} device.
	}
	\label{fig:SMFig-ballistic}
\end{figure}

\subsection{Internal loss rate dependence on drive power and bias current}\label{sec:kintib}

In addition to an increase in $\kappa_i$ for high drive powers as discussed in the main text, the internal loss rate of our circuit also depends on bias current.
%
We observe an increasing loss rate for increasing bias current, cf. Fig~\ref{fig:SMFig-lossrates}(b).
%
Possible origins of this phenomenon are low-frequency noise on the DC electronics, as this artificially widens the measured cavity resonance if the measurement time is greater than the inverse noise frequency.
%
Additionally, phase-slip events might occur at larger rates if the Josephson energy potential is tilted, as compared to zero bias current.

\begin{figure}[p]
	\centering
	\includegraphics[width=\linewidth]{chapter-gJJ-CPR/figs/SMFigure-lossrates}
	\caption{
		\textbf{Internal loss rate for increasing drive power (a) and bias current (b)}
		%
		An increase in $\kappa_i$ with drive power could be due to subgap states in the gJJ.
		%
		Increasing loss rate with bias current could originate from either low-frequency noise or phase slip events.
	}
	\label{fig:SMFig-lossrates}
\end{figure}

\begin{figure}
	\centering
	\includegraphics[width=\linewidth]{chapter-gJJ-CPR/figs/SMFigure-poweratJJ}
	\caption{
		\textbf{Voltage and current across the diffusive graphene Josephson junction.}
		%
		\textbf{(a,b)} Voltage across the JJ for varying gate voltage at reference power \textbf{(a)} and maximum drive power \textbf{(b)}.
		%
		\textbf{(c,d)} Current across the JJ for varying gate voltage at reference power \textbf{(c)} and maximum drive power \textbf{(d)}.
		%
		\textbf{(e,f)} Ratio of current across the JJ to DC-measured switching current for varying gate voltage at reference power \textbf{(e)} and maximum drive power \textbf{(f)}.
		%
		Note the different scales for the left and right column.
	}
	\label{fig:SMFigpoweratJJ}
\end{figure}


\subsection{Estimation of the fridge attenuation}\label{sec:attenuation}

We can estimate the attenuation of our RF input line by using the cryogenic HEMT as a calibrated noise source.
%
The HEMT noise power is given by
%
\begin{align}
P_{\rm HEMT}=10\log\left(\frac{k_B T_{\rm HEMT}}{\si{\milli\watt}}\right) + 10\log\left(\frac{\Delta f}{\si{\hertz}}\right)\ ,
\label{eq:HEMT}
\end{align}
%
with the Boltzmann constant $k_B$, the noise temperature of the HEMT $T_{\rm HEMT}=\SI{2}{\kelvin}$ as specified by the manufacturer and the measurement bandwidth $\Delta f=\SI{100}{\hertz}$.
%
The resulting noise power is $P_{\rm HEMT}=\SI{-175.59}{dBm}$.
%
Additionally, we can calculate the average background signal arriving at the VNA by averaging all $S_{11}$ traces in the areas off-resonant to the cavity, which leaves the background unaltered in power.
%
Doing so, we extract an average signal and standard deviation, which yields the signal-to-noise ratio at the VNA, $\text{SNR}_\text{VNA}=\SI{43.85}{\decibel}$, for a VNA output power of \SI{-20}{dBm}.
%
Assuming \SI{2}{\decibel} of cable loss between sample and HEMT, we arrive at an attenuation of \SI{111.74}{dB} of our VNA input line,

\subsection{Extraction of $I_s^{\rm DC}$ and $f_0$}\label{sec:extraction}

The DC switching current (Fig.~\ref{fig:figure1}(b,c)) is taken as the current at which $\partial V/\partial I_b$ is maximum, where $V$ is the measured voltage drop across the JJ.
%
Noise or interference on the DC lines could lead to a reduction of the measured $I_s^{\rm DC}$ compared to the true $I_c$.
%
To get a more accurate estimation of $I_c$ together with a good understanding of the noise sources, switching histograms are the preferred measurement method.
%
The necessary setup was however not available at the time of measurement.

To extract resonance frequency and loss rates from the RF data, we fit the reflection coefficient to the following model (cf. Ref.~\cite{bosmanBroadbandArchitectureGalvanically2015c} for a derivation):
%
\begin{align}
S_{11}(\omega) = -1+\frac{2\kappa_e}{\kappa+2i\Delta},
\end{align}
%
where $\kappa=\kappa_e+\kappa_i$ denoting the total, external and internal loss rates, respectively, and $\Delta=\omega-\omega_0$ with resonance frequency $\omega_0=2\pi f_0$.
%
The measured $S_{11}$ is usually distorted by a setup-related microwave background of the following shape:
\begin{align}
B(\omega) = \left(a+b\omega+c\omega^2\right)e^{i\left(a^\prime+b^\prime\omega\right)},
\end{align}
%
and with additional rotation by angle $\theta$ in the complex plane, the measured $S_{11}^\prime$ is:
\begin{align}
S_{11}^\prime(\omega)=B(\omega)\left(e^{i\theta}\left(S_{11}(\omega)+1\right)-1\right)
\end{align}
%
The origin of the microwave background and phase rotations are impedance mismatches in the wiring originating from various non-ideal circuit elements (e.g. connectors, attenuators, directional couplers, wirebonds).
%
Standing waves can form in some segments of the wiring which interfere with the measured signal, thus producing an oscillating measurement background.
%
To remove this background for the gate voltage sweeps (Fig.~\ref{fig:figure1}(d,e)), we pick the measurement trace at the CNP as the one with only background signal, as the RF resonance is extremely broad and effectively not present here.
%
We then divide the other traces by this trace, resulting in a much cleaner signal.
%
For measurements based on bias current sweeps, cf. Fig~\ref{fig:figure5}(a), we take the RF background as the $S_{11}$ trace at $I_b>I_s$.
%
Here, the JJ switched to the normal state and the RF resonance is not present in the measurement.
%
In order to remove RF background from the power dependence, we mask the regions in which there are resonances for the various powers and gate voltage setpoints, and average the remaining traces.
%
This way, we obtain a power and frequency map of the RF background, which we use for removing background signal from power traces, such as the one in Fig.~\ref{fig:figure4}(a).

\subsection{Derivation and validity of Eq.~\ref{eq:Pogorzalek}}\label{sec:validity}

\begin{figure}
	\centering
	\includegraphics[width=0.5\linewidth]{chapter-gJJ-CPR/figs/rfderivation}
	\caption{
		\textbf{Derivation of resonance frequency.}
		%
		We define the three impedances $Z_1$, $Z_2$ and $Z_q$ as seen from the CPW towards the input port, from the gJJ towards the CPW, and as the parallel circuit impedance.
	}
	\label{fig:rfderivation}
\end{figure}

We can derive an expression for the circuit resonance frequency depending on the other parameters by using the impedances defined in Fig.~\ref{fig:rfderivation}.
%
The circuit impedance as seen from the JJ towards the CPW, $Z_1$, the input impedance as seen from the CPW towards the input port, $Z_2$, and the overall parallel circuit impedance $Z_q$ are:
%
\begin{align}
Z_1 &= Z_0 \frac{Z_2+Z_0\tanh\gamma l}{Z_0+Z_2\tanh\gamma l} \\
Z_2 &= \left(\frac{1}{Z_{C_s}}+\frac{1}{Z_0}\right)^{-1} = \left(i\omega C_s+\frac{1}{Z_0}\right)^{-1} \\
Z_q &= \left(\frac{1}{Z_{JJ}}+\frac{1}{Z_1}\right)^{-1} = \left(\frac{1}{i\omega L_J}+\frac{1}{Z_1}\right)^{-1}\ ,
\end{align}
%
with the CPW length $l$, the complex CPW loss per unit length $\gamma=\alpha+i\beta$, and the transmission line impedance $Z_0$.
%
Assuming negligible losses in the CPW on resonance, $\gamma l\approx i\beta l = i\pi\omega_0/\omega_r$, i.e. the CPW only acts as a phase shifter.
%
The resonance condition of the above circuit is for the imaginary part of the admittance $Y=1/Z_q$ to be zero, which yields
%
\begin{align}
0 = \Im \left[ \frac{1}{i\omega_0 L_J} + \frac{1}{Z_0}\frac{Z_0+iZ_2\tan\left(\pi\omega_0/\omega_r\right)}{Z_2+iZ_0\tan\left(\pi\omega_0/\omega_r\right)}\right]
\label{eq:SolAnalytical}
\end{align}
%
The solution of this equation for $\omega_0$ is plotted as solid line in Fig.~\ref{fig:SMval}(a).
%
We can approximate the above by a similar method as the authors of Refs.~\cite{wallquistSelectiveCouplingSuperconducting2006a,wustmannParametricResonanceTunable2013,pogorzalekHystereticFluxResponse2017}:
%
Assuming a large shunt capacitance at the input, such that $Z_2\approx 0$ and expanding the tangent, we arrive at the expression stated in Eq.~\ref{eq:Pogorzalek} which is plotted as dashed line in Fig.~\ref{fig:SMval}(a).
%
This assumption is justified since $C_s\approx\SI{27}{\pico\farad}$ for our devices, such that both $Z_2\approx \SI{0.2}{\ohm} \ll Z_0=\SI{50}{\ohm}$.
%
We find that for all values of $L_J$, including the range in our experiments, the approximation differs by no more than \SI{1}{\percent} from the analytical solution, regardless of JJ to CPW impedance, cf. Figs.~\ref{fig:SMval}(b-c).

\subsection{Circuit parameters}

\begin{figure}
	\centering
	\includegraphics[width=\linewidth]{chapter-gJJ-CPR/figs/SMFigure-validity}
	\caption{
		\textbf{Validity of Eq.~\ref{eq:Pogorzalek}.}
		%
		The equation used for modeling $f_0$ of the microwave circuit is an approximation to the true value.
		%
		\textbf{(a)} Resonance frequency as calculated using the  (solid) and using Eq.~\ref{eq:Pogorzalek} (dashed) for increasing Josephson inductance.
		\textbf{(b)} Relative difference between the two resonance frequencies.
		%
		\textbf{(c)} Impedance of the JJ (solid) and CPW impedance (dashed).
		%
		The approximation is in acceptable agreement with the full model regardless of $Z_J/Z_0$.
	}
	\label{fig:SMval}
\end{figure}

The values of the bare CPW resonator frequency and inductor are listed in Tab.~\ref{tab:frLr}.

\begin{table}[!h]
	\centering
	\caption{\textbf{Circuit parameters as extracted for the various measurements}}
	\begin{tabular}{ccccc}
		\hline\hline
		& & Diffusive JJ & \multicolumn{2}{c}{Ballistic JJ}  \\
		& Temperature & \SI{15}{\milli\kelvin} & \SI{15}{\milli\kelvin} & \SI{1}{\kelvin} \\
		\hline
		\multirow{ 2}{*}{$V_g$ sweep} & $f_{\lambda/2}$ (\si{\giga\hertz}) & $8.257$ & $8.339$ & $8.280$ \\
		& $L_r$ (\si{\nano\henry}) & $4.000$ & $2.757$ & $5.608$ \\
		\hline
		\multirow{ 2}{*}{$I_b$ sweep} & $f_{\lambda/2}$ (\si{\giga\hertz}) & $8.253$ & $8.287$ & $-$ \\
		& $L_r$ (\si{\nano\henry}) & $3.857$ & $3.622$ & $-$ \\
		\hline\hline
	\end{tabular}
	\label{tab:frLr}
\end{table}



\subsection{Fitting procedure for extracting $\beta$ from power dependence}\label{sec:SMduffing}

Following the method described in Ref.~\cite{schmidtCurrentDetectionUsing2020}, the equation of motion of the amplitude field $\alpha(t)$ of a resonator with weak anharmonicity $\beta$ written in the frame rotating with the drive $S_{\rm in}$ is given by Eq.~\ref{eq:Duffing-EOM}, from which the steady-state solution $\partial\alpha_0/\partial t=0$ results in the polynomial function
% 
\begin{align}
\beta^2 \alpha_0^6 + 2\Delta\beta\alpha_0^4 + \left(\Delta^2+\frac{\kappa^2}{4}\right)\alpha_0^2 - \kappa_\text{e} \abs{S_{\rm in}}^2 = 0\ ,
\label{eq:polynom}
\end{align}
%
which we can solve and use to calculate the expected reflection coefficient as our model,
\begin{align}
S_{11}=-1-\frac{\sqrt{\kappa_e}}{S_{\rm in}}\alpha_0\ .
\label{eq:S11anh}
\end{align}
%
to fit the measurement data.
%
We reduce the number of free parameters of this function from five to two by fixing $\omega_0$ and $\kappa_e$ as the values extracted at lowest drive power and calculating $S_{\rm in}$ from the fridge attenuation, see Supplementary Section Sec.~\ref{sec:attenuation}.
%
The remaining parameters are $\beta$ and $\kappa_i$, where the internal loss rate can in fact depend on the drive power, $\kappa_i=\kappa_i(S_{\rm in})$.
%
Fixing the loss rate to be constant throughout the fit does not lead to a good fit to the data, as shown in Fig.~\ref{fig:SMpower}.
%
Our algorithm first fits the measured data to return constant $\beta$ and $\kappa_i$, and uses these as initial values for a fit to extract the power dependent loss rate, shown in Fig.~\ref{fig:SMFig-lossrates}.

\begin{figure}
	\centering
	\includegraphics[width=\linewidth]{chapter-gJJ-CPR/figs/SMFigure-power}
	\caption{
		\textbf{Anharmonicity fit assuming different cases for $\kappa_i$.}
		%
		Fixing $\kappa_i$ to be the value at lowest drive power (first column) results in significantly worse fit than introducing it as constant, but free parameter (second column).
		%
		However, best agreement between data and model is reached when introducing nonlinear damping (third column and Fig.~\ref{fig:figure4}).
		%
		Linecuts and colors correspond to the ones in Fig.~\ref{fig:figure4}.
	}
	\label{fig:SMpower}
\end{figure}

\subsection{Fitting procedure for extracting $\tau$ from bias-current dependence}\label{sec:fitbiascurrent}

\begin{figure}
	\centering
	\includegraphics[width=\linewidth]{chapter-gJJ-CPR/figs/SMFigure-influence}
	\caption{
		\textbf{Predicted influence of the junction transparency on the bias current dependence.}
		%
		\textbf{(a)} CPR for various $\tau=0$ (solid), $\tau=0.5$ (dashed) and $\tau=1.0$	(dash-dotted).
		%
		\textbf{(b-c)} Josephson inductance \textbf{(b)} and resonance frequency \textbf{(c)} supercurrent for	same transparencies as in \textbf{(a)}.
		%
		Increased junction transmission leads to forward skewing of the CPR, thus a reduced slope and higher Josephson inductance, which in turn reduces the resonance frequency and increases the tuning.
	}
	\label{fig:SMinfluence}
\end{figure}

Without any knowledge on the junction transparency $\tau$, fitting data of a CPW cavity with JJ exhibiting a potentially nonsinusoidal CPR can lead to significant deviations from the true circuit parameters.
%
In Fig.~\ref{fig:SMtau}, we demonstrate the effect of $\tau$ on the extracted circuit parameters:
%
If the JJ shorting the CPW to ground has a forward skewed CPR, i.e. $\tau>0$, and Eq.~\ref{eq:Pogorzalek} is used to fit this data (dashed lines in Fig.~\ref{fig:SMtau}) under the assumption of a JJ with purely sinusoidal CPR, the extracted values (solid lines) for $f_r$ and $L_r$ will be too small compared with the true values, while $I_c$ will appear to be larger than expected, with consequently lower $L_J$.
%
Naively, one would assume the RF measurement to result in $L_J$ to be larger than the value expected from DC measurements due to the increased inverse CPR slope for identical $I_c$.
%
Nonetheless, the fit model accounting for the resonator tunability results in the opposite behavior.

To fit the bias current dependence data for extracting $\tau$, we first keep $\tau$ fixed and compare the fit residuals for various $\tau\in[0,1]$.
%
In a second step, we set $\tau$ as additional free parameter with the best results for $f_r$, $L_r$ and $I_c$, and set the initial value of $\tau$ as the one with previously determined minimum reduced $\chi^2$.
%
The fit converges more reliably than by setting $\tau$ as free parameter from the start, since this way the fit does not get trapped in a local minimum.

Both current and frequency noise, however, can result in artificial global minima in $\chi^2(\tau)$.
%
For a critical current of \SI{2}{\micro\ampere} and bias currents up to $0.9I_c$, already $\sigma_I\geq\SI{1}{\nano\ampere}$ is enough to significantly throw off the fit for low values of $\tau$.
%
Frequency noise of $\sigma_f<\SI{1}{\mega\hertz}$ has no influence on the fitting algorithm.
%
As the responsivity $\partial f_0/\partial I_b$ increases with $I_b$, so does the frequency noise for a fixed current noise.
%
From the average data fluctuations and comparing these with the expected $\sigma_f(\sigma_I)$, we estimate $\sigma_I<\SI{1}{\nano\ampere}$ in our setup.
%
We therefore conclude that our fitting algorithm should present a reliable way of extracting $\tau$. 

%\begin{figure}
%	\centering
%	\includegraphics[width=\linewidth]{chapter-gJJ-CPR/figs/SMFigure-tauparams}
%	\caption{
%		\textbf{Influence of $\tau$ on extracted fit parameters with Eq.~\ref{eq:Pogorzalek}.}
%		Increasing $\tau$ (dashed line in \textbf{(a)}) results in a larger $L_J$ (dashed line in \textbf{(e)}), while the other fit parameters (dashed lines in \textbf{(b-d)}) remain constant.
%		%
%		In order for a fit model using Eq.~\ref{eq:Pogorzalek} under the assumption of a sinusoidal CPR (solid lines) to reproduce the data (points), significant deviations from the true parameters occur.
%		%
%		Specifically, the fit model returns a larger critical current than expected which leads to a reduced calculated $L_J$, even though the real $L_J$ increases with $\tau$.
%	}
%	\label{fig:SMtau}
%\end{figure}
%
%\begin{figure}
%	\centering
%	\includegraphics[width=\linewidth]{chapter-gJJ-CPR/figs/SMFigure-noise}
%	\caption{
%		\textbf{Influence of $\tau$ and noise on fit result.}
%		\textbf{(a)} Assuming a sinusoidal CPR for Eq.~\ref{eq:Pogorzalek} to fit data originating from a forward skewed CPR results in minimum fit residuals for $\tau=0$ (solid line).
%		%
%		Assuming the correct $\tau$ however results in minimum residuals (dashed line).
%		%
%		\textbf{(b-d)} Both current and frequency noise can significantly throw off the fitting algorithm by creating artificially global residual minima.
%		%
%		Plotted are calculations for $I_c=\SI{2}{\micro\ampere}$.
%		%
%		We estimate $\sigma_I<\SI{1}{\nano\ampere}$ in our setup.
%	}
%	\label{fig:SMres}
%\end{figure}

\references{dissertation}

