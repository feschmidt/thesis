\chapter*{Summary}
\addcontentsline{toc}{chapter}{Summary}
\setheader{Summary}

{%\color{title}

This thesis investigates fundamental properties of Josephson junctions embedded in microwave circuits, and an application arising from this hybrid approach.
%
We used the versatility of superconducting coplanar DC bias cavities to extract previously inaccessible information on phase coherent and subgap mechanisms of graphene Josephson junctions.

\noindent \newline
Chapter~\ref{chap:intro} gives an introduction to the promising field of Josephson field effect transistors, among which graphene junctions show promise for future improvements in quantum computation.
%
Together with an overview of the Josephson effect in superconducting-semiconducting systems, we introduce the concept of coplanar DC bias cavities for probing Josephson junctions.

\noindent \newline
In chapter~\ref{chap:experiment}, we describe the experimental methods developed for carrying out the subsequent measurements.
%
We include details on fabrication, material properties and measurement setup.

\noindent \newline
Results of graphene Josephson junctions embedded in DC bias microwave resonators are presented in chapters~\ref{chap:gJJ} and \ref{chap:gJJ-CPR}.
%
By following the resonance frequency and losses of the circuit, we are able to extract the junctions' Josephson inductance and subgap resistance.
%
Studying the nonlinear power and bias current response reveals further information on the underlying loss mechanisms and current phase relation.

\noindent \newline
We turn to an application of our hybrid bias cavity -- Josephson junction devices to detect small, low-frequency currents in chapter~\ref{chap:currentdetection}.
%
Our device is competitive with state-of-the-art techniques for microwave radiation detection and, with minor modifications, should be able to outperform existing technologies by orders of magnitude.

\noindent \newline
Finally, we conclude the presented work in chapter~\ref{chap:conclusion} and provide an outlook on potential future research.

}
\afterpage{\pagecolor{none}}

\chapter*{Samenvatting}
\addcontentsline{toc}{chapter}{Samenvatting}
\setheader{Samenvatting}

%% {\selectlanguage{dutch}


\afterpage{\pagecolor{none}}

\chapter*{Zusammenfassung}
\addcontentsline{toc}{chapter}{Zusammenfassung}
\setheader{Zusammenfassung}

%% {\selectlanguage{german}


\afterpage{\pagecolor{none}}