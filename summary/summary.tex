\chapter*{Summary}
\addcontentsline{toc}{chapter}{Summary}
\setheader{Summary}

This thesis investigates fundamental properties of Josephson junctions embedded in microwave circuits, and an application arising from this hybrid approach.
%
We used the versatility of superconducting coplanar DC bias cavities to extract previously inaccessible information on phase coherent and subgap mechanisms of graphene Josephson junctions.

\noindent \newline
Chapter~\ref{chap:intro} gives an introduction to the technology of Josephson field effect transistors, among which graphene junctions show promise for future improvements in quantum computation.
%
Together with an overview of the Josephson effect in superconducting-semiconducting systems, we introduce the concept of coplanar DC bias cavities for probing Josephson junctions at gigahertz frequencies.

\noindent \newline
In chapter~\ref{chap:experiment}, we describe the experimental methods developed for carrying out the subsequent measurements.
%
We include details on fabrication, material properties and measurement setup.

\noindent \newline
Results of graphene Josephson junctions embedded in DC bias microwave resonators are presented in chapters~\ref{chap:gJJ} and \ref{chap:gJJ-CPR}.
%
By following the resonance frequency and losses of the circuit, we are able to extract the junctions' Josephson inductance and subgap resistance.
%
Studying the nonlinear power and bias current response reveals further information on the underlying loss mechanisms and current phase relation.

\noindent \newline
We turn to an application of our hybrid bias cavity -- Josephson junction devices to detect small, low-frequency currents in chapter~\ref{chap:currentdetection}.
%
Our device is competitive with state-of-the-art techniques for microwave radiation detection and, with minor modifications, should be able to outperform existing technologies by orders of magnitude.

\noindent \newline
Finally, we conclude the presented work in chapter~\ref{chap:conclusion} and provide an outlook on potential future research.

%\afterpage{\pagecolor{none}}

\chapter*{Samenvatting}
\addcontentsline{toc}{chapter}{Samenvatting}
\setheader{Samenvatting}

{%\color{title}
\selectlanguage{dutch}

Dit proefschrift onderzoekt fundamentele eigenschappen van Josephson-juncties ingebed in microgolfcircuits, en een toepassing die voortkomt uit deze hybride benadering.
%
We gebruikten de veelzijdigheid van supergeleidende coplanaire DC-bias-holtes om eerder ontoegankelijke informatie over fase-coherente en subgap-mechanismen van grafeen Josephson-juncties te extraheren.

\noindent\newline
Hoofdstuk~\ref{chap:intro} geeft een inleiding tot de technologie van Josephson veldeffecttransistors, waaronder grafeenknooppunten die veelbelovend zijn voor toekomstige verbeteringen in kwantumberekening.
%
Samen met een overzicht van het Josephson-effect in supergeleidende halfgeleidende systemen introduceren we het concept van coplanaire DC-bias-holtes voor het onderzoeken van Josephson-juncties op gigahertz-frequenties.

\noindent\newline
In hoofdstuk~\ref{chap:experiment} beschrijven we de experimentele methoden die ontwikkeld zijn voor het uitvoeren van de volgende metingen.
%
We nemen details op over fabricage, materiaaleigenschappen en meetopstellingen.

\noindent\newline
Resultaten van grafeen Josephson-juncties ingebed in DC-bias microgolfresonatoren worden gepresenteerd in de hoofdstukken~\ref{chap:gJJ} en \ref{chap:gJJ-CPR}.
%
Door de resonantiefrequentie en verliezen van het circuit te volgen, kunnen we de Josephson-inductantie en subgap-weerstand van de juncties extraheren.
%
Het bestuderen van de niet-lineaire kracht en de vertekening van de huidige respons onthult meer informatie over de onderliggende verliesmechanismen en de huidige fase-relatie.

\noindent\newline
We wenden ons tot een toepassing van onze hybride bias-holte -- Josephson-junctie-apparaten om kleine, laagfrequente stromen te detecteren in hoofdstuk ~\ref{chap:currentdetection}.
%
Ons apparaat is concurrerend met de allernieuwste technieken voor microgolfstralingdetectie en zou, met kleine aanpassingen, in staat moeten zijn om bestaande technologieën te overtreffen in grootteorde.

\noindent\newline
Ten slotte sluiten we het gepresenteerde werk af in hoofdstuk~\ref{chap:conclusion} en geven een blik op potentieel toekomstig onderzoek.

}

%\afterpage{\pagecolor{none}}

\chapter*{Zusammenfassung}
\addcontentsline{toc}{chapter}{Zusammenfassung}
\setheader{Zusammenfassung}

{%\color{title}
\selectlanguage{german}
In dieser Dissertation untersuchen wir grundlegende Eigenschaften von in Mikrowellenschaltkreisen eingebetteten Josephson-Kontakten, sowie eine Anwendung, die aus diesem hybriden Ansatz folgt.
%
Wir nutzten die Vielseitigkeit von supraleitenden koplanaren Gleichstromresonatoren um bisher nicht zugängliche Information über Phasenkohärente und Subgapwiderstände von Graphen Josephsonkontakten zu extrahieren.

\noindent \newline
Kapitel~\ref{chap:intro} beinhaltet eine Einleitung in die Technologie der Josephson-Feldeffekttransistoren, unter denen unter anderem Graphen vielversprechende Ansätze für zukünftige Verbesserungen für Quantencomputer zeigt.
%
Gemeinsam mit einem Überblick über den Josephsoneffekt in supra-halbleitenden Systemen, legen wir einen Überblick auf koplanare Gleichstromresonatoren für die Untersuchung von Graphenkontakten bei Frequenzen im Gigahertzbereich.

\noindent \newline
In Kapitel~\ref{chap:experiment} beschreiben wir die experimentellen Methoden, welche wir für die Durchführung der folgenden Experimente entwickelten.
%
Dies beinhaltet Details zur Probenherstellung, Materialeigenschaften und dem Messaufbau.

\noindent \newline
Wir präsentieren Ergebnisse von in Gleichstrommikrowellenresonatoren eingebetteten Graphen Josephsonkontakten in Kapiteln~\ref{chap:gJJ} und \ref{chap:gJJ-CPR}.
%
Durch Messung der Resonanzfrequenz und Verluste des Schaltkreises konnten wir die Josephsoninduktivität und den Subbandlückenwiderstand quantifizieren.
%
Weitere Hintergründe der zugrunde liegenden Mechanismen der Mikrowellenverluste und der Strom-Phasenbeziehung konnten durch Analyse der nichtlinearen Leistungs- und Stromabhängigkeit des Schwingkreises gezogen werden.

\noindent \newline
Im Gegensatz zu den vorangegangenen grundlegenden Untersuchungen beschäftigen wir uns in Kapitel~\ref{chap:currentdetection} mit der Anwendung der Kombination aus Gleichsstrombiasresonator und Josephsonkontakt zur Detektion kleiner, niedrigfrequenter Ströme.
%
Unser Gerät ist mit den neuesten Techniken zur Detektion von Mikrowellenstrahlung konkurrenzfähig und sollte mit geringfügigen Änderungen in der Lage sein, vorhandene Technologien um Größenordnungen zu übertreffen.

\noindent \newline
Abschließend schließen wir die vorgestellte Arbeit in Kapitel~\ref{chap:conclusion} ab und geben einen Ausblick auf mögliche zukünftige Forschungen.

}

%\afterpage{\pagecolor{none}}