\chapter{Collection of DC data of gJJs and gSQUIDs}
%\chapter{Miscellaneous data on gJJs embedded in microwave circuits}
\label{chap:gJJmisc}

%\epigraph[0pt]{
%	Something.
%}{Someone}
%
%
%\begin{abstract}
%	\color{title}
%	Smart sounding abstract
%\end{abstract}

%% Start the actual chapter on a new page.
\afterpage{\pagecolor{none}}\newpage

References for Fiske steps~\cite{coonJosephsonAcStep1965,changFiskeStepsJosephson1983,krasnovFiskeStepsIntrinsic1999,kimFiskeStepsStudied2005,yabukiSupercurrentVanWaals2016b,liHighQualityEpitaxialMgB2017}

References for Shapiro steps~\cite{shapiroJosephsonCurrentsSuperconducting1963,kautzNoiseChaosJosephson1996,tinkhamIntroductionSuperconductivity1996,heerscheBipolarSupercurrentGraphene2007a,leeUltimatelyShortBallistic2015,shellyExistenceShapiroSteps2020,larsonZerobiasCrossingsPeculiar2020}

Additional data
TODO: herodevice
We observed lots of oscillations in DC, but no RF signal
No RF cavity visible because it's dead due to the normal resistance around the SQUID
BUT: junction itself is probing the electric envirnoment and sees a plasma resonance at the cavity --> QUCS sims

Figures:
\begin{itemize}
	\item optical of devices
	\item IVCs v Vg
	\item Rn v Vg
	\item IVC oscillation steps v Vg
	\item Shapiros
	\item RF simulation QUCS: resonance f0 and Q as a function of normal resistance around the JJ
\end{itemize}

%\clearpage
%\references{dissertation}

