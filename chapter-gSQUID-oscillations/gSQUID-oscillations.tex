\chapter{Spectroscopy of a graphene SQUID}
\label{chap:herodevice}

\epigraph[0pt]{
	Something.
}{Someone}


\begin{abstract}
	Smart sounding abstract
\end{abstract}

%% Start the actual chapter on a new page.
\newpage


TODO: herodevice
We observed lots of oscillations in DC, but no RF signal
No RF cavity visible because it's dead due to the normal resistance around the SQUID
BUT: junction itself is probing the electric envirnoment and sees a plasma resonance at the cavity --> QUCS sims
Fig1: the device
Fig2: Gate dependent IVs + VIs+ step heights
Fig3: Field-dependent IVs + VIs + step heights
Fig4: Shapiro + analysis of step heights
Fig5: RF simulation QUCS: resonance f0 and Q as a function of normal resistance around the JJ


\begin{figure}
	\centering
	\includegraphics[]{example-image}
	\caption{
		\textbf{Graphene SQUID embedded in a DC bias microwave cavity.}
		\textbf{A,} Optical image of cavity.
		\textbf{B,} Zoom-in of SQUID.
		\textbf{C,} Circuit schematic.
		\textbf{D,} More detailed circuit schematic of junction area.
	}
	\label{fig:placeholder}
\end{figure}

\references{dissertation}

