\newchapstyle
\chapter{Conclusion}
\label{chap:conclusion}

\afterpage{\pagecolor{none}}\newpage

\section{The results of this thesis}
In this thesis, we presented studies on hybrid Josephson junction -- DC bias microwave circuits.
%
We studied both the fundamental properties of graphene Josephson junctions at microwave frequencies, and used the circuit which we developed for detection of very small currents.

Chapter~\ref{chap:gJJ} featured the integration of a graphene Josephson junction into a superconducting microwave circuit.
%
Using an analytical circuit model and the previously calibrated parameters of a DC bias cavity allowed us to quantitatively extract the Josephson inductance from microwave measurements.
%
This showed significant deviations between the value extrapolated from DC measurements, consistent with a skewed current phase relation.
%
Analysis of the subgap resistance of the junction from the microwave losses of the resonance suggests that graphene Josephson junctions are a feasible platform for building gate-tunable microwave qubits.

In chapter~\ref{chap:gJJ-CPR}, we took a closer look at the underlying physics of the Josephson inductance of these graphene devices, namely the current-phase relation.
%
The high-frequency response of the devices, together with circuit calibration, allows for a direct way of extracting the Josephson inductance of our junctions.
%
As in chapter~\ref{chap:gJJ}, this value does not agree with the one estimated from the DC measured switching current of any of the junctions, with some DC values being larger and others being smaller than the ones observed in the microwave regime.
%
By analysing the DC current response of the microwave resonance, we found low-frequency current noise to have reduced the DC measured switching current significantly.
%
The noise source is most likely poor isolation between the battery powered DC electronics and mains powered high frequency equipment via a shared ground at room temperature.
%
The remaining deviations can be attributed to a forward-skewed current phase relation.
%
We were able to quantify this skew, and with it the resulting anharmonicity of the Josephson energy potential, which is reduced to the case of a purely sinusoidal CPR.

We finally turned to a sensing application of our combined transmission line resonators and Josephson junctions in chapter~\ref{chap:currentdetection}.
%
Inspired by kinetic inductance parametric upconverters for radiation detection, we used the responsivity of the resonance frequency of our devices to bias current to detect small, low-frequency current modulations.
%
To simplify fabrication, we chose a single-layer constriction Josephson junction fabricated in the same step as the base layer of an aluminum film instead of graphene Josephson junctions (which would also have been feasible for this measurement).
%
We were able to formulate an analytical model using input-output theory, which compared well with the measured data as a function of bias current set point, frequency detuning and and drive power.
%
With a minimum sensitivity of \SI{8.9}{\pico\ampere\per\hertz\tothe{1/2}} and a tunability of more than \SI{100}{\mega\hertz}, the device is compatible with state-of-the-art techniques.
%
We then used these results to extrapolate the sensitivity of an optimized design, featuring a Josephson transmission line, with the center conductor of the DC bias cavity replaced by unit cells consisting of short pieces of linear inductors and Josephson junctions.
%
The same scheme as for the measured device would enable current sensitivities as low as \SI{50}{\femto\ampere\per\hertz\tothe{1/2}}, and drop another \SI{20}{\decibel} to \SI{5}{\femto\ampere\per\hertz\tothe{1/2}} by using in-built quantum-limited Josephson parametric amplification.

\section{The road ahead}

%\subsection{Josephson field effect transistor}

%\subsection{Graphene for superconducting MW circuits}

As we have seen in Chapters~\ref{chap:gJJ} and \ref{chap:gJJ-CPR}, with current technology and fabrication, encapsulated graphene Josephson junctions are not an immediately scalable option for quantum Josephson devices.
%
This is mainly due to the remaining high intrinsic losses, which are likely limiting graphene transmon qubits such as the ones in Refs.~\cite{krollMagneticFieldCompatible2018,wangCoherentControlHybrid2019}.

The two main issues that need to be addressed in order to improve this are (1) disorder at the graphene-metal interface and (2) the cumbersome BN encapsulation:
%
Dry-etching using \ce{CHF3 + O2} in order to make galvanic contacts with the graphene layer can add surface defects in the silicon area close to the junction, providing a number of loss channels.
%
It would therefore be beneficial to pre-pattern graphene or at least remove residual stacks, but this significantly adds to fabrication complexity.
%
BN encapsulation on the other hand remains non-scalable as long as it cannot be done on a wafer scale.
%
In particular, the hydrogen bubbles at BN/G interfaces limit device usability, since annealing in forming gas at \SI{400}{\celsius} degrades the MW properties of the superconducting layer used for the MW circuitry and can potentially lead to surface defects in the silicon layer.
%
Recent work by Sonntag \textit{et al.}~\cite{sonntagExcellentElectronicTransport2020} on BN crystals grown at atmospheric pressure shows promise for large-scale BN deposition, which could advance the field in this regard.

A more viable route in the immediate future, which could be investigated in parallel, are unencapsulated Josephson junctions based on CVD-grown single layer graphene with top contacts.
%
A possible fabrication scheme for this process could feature the following steps:
%
(1) Pre-pattern all metal layers including bottom gate, (2) cover bottom gate by local sputtering of dielectric, (3) transfer SLG on wafer scale, (4) pattern SLG using \ce{O2} plasma, (5) contact SLG using Ti/Al evaporation.
%
The use of CVD graphene would have the significant advantage of wafer scale processing, enabling rapid prototyping and significantly larger throughput than both manual and automatic exfoliation and heterostructure assembly of graphene and boron nitride.
%
It remains to be seen if the device quality would be sufficient for coherent devices.

%\subsection{Usefulness of current bias cavities}

From the perspective of DC bias cavities, there are a few possible follow-up projects:

The DC bias cavities could be improved further, as they are currently likely limited by dielectric losses originating from depositing the shunt capacitor dielectric.
%
As described in chapter~\ref{chap:experiment}, side-coupled coplanar resonators made from the same NbTiN film as our DC bias cavities showed internal quality factors more than ten times higher our best devices.
%
Dielectric losses at the metal-substrate interface are therefore not to blame.
%
Similarly, radiative losses are unlikely the dominant source due to the coplanar geometry.
%
However, the DC bias cavity fabrication in its current state requires three steps (compared to one for $\lambda/4$ resonators) and dielectric deposition on the entire superconducting surface with subsequent patterning.
%
Switching dielectrics to low-temperature deposited low-loss materials such as aluminum oxide or silicon could enable much lower internal loss rates and more coherent devices~\cite{adamyanTunableSuperconductingMicrostrip2016}.

Future research could examine our circuit layouts for various applications:
%
For one, the Josephson transmission line resonator which we proposed in chapter~\ref{chap:currentdetection} should be straightforward to build, since it requires no design changes to the existing recipe.
%
This would put our analytical model to the test, and could result in record low levels of current sensitivities.

Alternatively, instead of current-biasing our devices, voltage-biased Josephson junctions inside coplanar waveguide cavities have recently shown to exhibit coherent radiation, and were hence dubbed Josephson lasers~\cite{chenRealizationSingleCooperpairJosephson2014c,cassidyDemonstrationAcJosephson2017e}.
%
The emission frequency of these recent implementations lies in the gigahertz range and is highly coherent, which makes them attractive for on-chip microwave sources.
%
However, the existing approaches are limited to a fixed frequency with only a few~\si{\percent} of frequency tuning.
%
Additionally, the circuit design relies on introducing DC leads to a cavity at voltage nodes, which can be affected by lithographic errors.
%
Using the shunt capacitor DC bias cavities would allow for arbitrary frequency targeting:
%
Since the input port of the DC and high frequency signals are identical, there are no constraints on the length of the circuit.
%
Building a low-frequency resonator simply by extending the transmission line length, shorted by a Josephson junction to ground, could result in a megahertz Josephson laser, enabling a significantly larger frequency span than existing devices by using the cavity overtones in addition to the fundamental mode.

%current detection~\cite{kherKineticInductanceParametric2016,kherSuperconductingNonlinearKinetic2017}
%


The road ahead might even feature a renaissance of all-superconducting DC circuits for computing applications.
%
With cryotrons still being pursued both for memory applications and training neural networks~\cite{toomeyBridgingGapNanowires2019,onenDesignCharacterizationSuperconducting2020}, and the discovery of field effect in all-superconducting devices~\cite{desimoniMetallicSupercurrentFieldeffect2018,paolucciMagnetotransportExperimentsFully2019}, we can only wait and see what is to come.


%\clearpage
%\references{dissertation}