\chapter{Conclusion}
\label{conclusion}

This is a concluding chapter explaining the scientific and technical
implications for society of the research findings in considerable detail.

\section{Future of graphene}

With current technology and fabrication, encapsulated graphene Josephson junctions are not a scalable option for quantum Josephson devices.
The following pitfalls would need to be solved in order to improve this:

- Dry-etching using CHF3+O2 in order to make galvanic contacts with the graphene layer can add surface defects in the silicon area close to the junction, providing a number of loss channels.
It would therefore be beneficial to pre-pattern graphene or at least remove residual stacks, but this significantly adds to fabrication complexity
- BN encapsulation remains non-scalable as long as it cannot be done on a wafer scale.
Especially the hydrogen bubbles at BN/G interfaces limit device usability, since annealing in forming gas at \SI{400}{\celsius} degrades the RF properties of the superconducting layer used for the RF circuitry and can potentially lead to surface defects in the silicon layer

A more viable route could be Josephson junctions based on CVD-grown single layer graphene.
It remains to be seen if the device quality would be sufficient, since AlOx JJs are also not ballistic.
A possible fabrication scheme could look like this:

1. Pre-pattern all metal layers including bottom gate
1. Cover bottom gate by local sputtering of dielectric
1. Transfer SLG on wafer scale
1. Pattern SLG using O2 plasma
1. Contact SLG using Ti/Al evaporation

\section{Usefulness of current bias cavities}

Josephson laser
