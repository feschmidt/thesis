\newchapstyle
\chapter{Conclusion}
\label{chap:conclusion}

\afterpage{\pagecolor{none}}\newpage
This is a concluding chapter explaining the scientific and technical
implications for society of the research findings in considerable detail.

\section{Josephson field effect transistor}

Dayem bridge~\cite{paolucciMagnetotransportExperimentsFully2019}
All metallic~\cite{desimoniMetallicSupercurrentFieldeffect2018}
\section{Graphene for superconducting MW circuits}

As we have seen in Chapters~\ref{chap:gJJ} and \ref{chap:gJJ-CPR}, with current technology and fabrication, encapsulated graphene Josephson junctions are not a scalable option for quantum Josephson devices.
%
This is mainly due to the remaining high intrinsic losses, which are likely limiting graphene transmon qubits such as the ones in Refs.~\cite{krollMagneticFieldCompatible2018,wangCoherentControlHybrid2019}.

The following pitfalls would need to be addressed in order to improve this:
\begin{description}
	\item[Graphene-metal interface] Dry-etching using \ce{CHF3 + O2} in order to make galvanic contacts with the graphene layer can add surface defects in the silicon area close to the junction, providing a number of loss channels.
	%
	It would therefore be beneficial to pre-pattern graphene or at least remove residual stacks, but this significantly adds to fabrication complexity
	\item[BN encapsulation] BN encapsulation remains non-scalable as long as it cannot be done on a wafer scale.
	%
	Especially the hydrogen bubbles at BN/G interfaces limit device usability, since annealing in forming gas at \SI{400}{\celsius} degrades the RF properties of the superconducting layer used for the RF circuitry and can potentially lead to surface defects in the silicon layer.
\end{description}

A more viable route could be Josephson junctions based on CVD-grown single layer graphene.
%
It remains to be seen if the device quality would be sufficient, since \ce{AlO_x} JJs are also not ballistic.
%
A possible fabrication scheme could look like this:
%
\begin{itemize}
	\item Pre-pattern all metal layers including bottom gate
	\item Cover bottom gate by local sputtering of dielectric
	\item Transfer SLG on wafer scale
	\item Pattern SLG using \ce{O2} plasma
	\item Contact SLG using Ti/Al evaporation
\end{itemize}


\section{Usefulness of current bias cavities}

Josephson laser~\cite{chenRealizationSingleCooperpairJosephson2014c,cassidyDemonstrationAcJosephson2017e}, current detection~\cite{kherKineticInductanceParametric2016,kherSuperconductingNonlinearKinetic2017}, JPA~\cite{hoeomWidebandLownoiseSuperconducting2012d}

Improved dielectrics for DC shunt bias~\cite{adamyanTunableSuperconductingMicrostrip2016}

Outlook on other architectures:
%
Direct leads at voltage nodes of $\lambda$ and $\lambda/2$ resonators~\cite{chenIntroductionDcBias2011a,liApplyingDirectCurrent2013}.
%
Inductive coupling~\cite{vissersFrequencytunableSuperconductingResonators2015b}.
%
Fractals~\cite{mahashabdeFastTunableHigh2020}

%\clearpage
%\references{dissertation}