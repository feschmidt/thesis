\chapter{Introduction}
\label{chap:intro}



%% The '0pt' option ensures that no extra vertical space follows this epigraph,
%% since there is another epigraph after it.
\epigraph[0pt]{
    Nature and nature's laws lay hid in the night; \\
    God said `Let Newton be!' and all was light.
}{Alexander Pope}

\begin{abstract}
Lorem ipsum dolor sit amet, consectetur adipisicing elit, sed do eiusmod tempor incididunt ut labore et dolore magna aliqua. Ut enim ad minim veniam, quis nostrud exercitation ullamco laboris nisi ut aliquip ex ea commodo consequat. Duis aute irure dolor in reprehenderit in voluptate velit esse cillum dolore eu fugiat nulla pariatur. Excepteur sint occaecat cupidatat non proident, sunt in culpa qui officia deserunt mollit anim id est laborum.
\end{abstract}

%% Start the actual chapter on a new page.
\newpage

\section{The Josephson Field Effect Transistor}

\dropcap{S}{ince} a dissertation is a substantial document, it is convenient to break it up into smaller pieces. In this template we therefore give every chapter its own file. The chapters (and appendices) are gathered together in \texttt{dissertation.tex}, which is the master file describing the overall structure of the document. \texttt{dissertation.tex} starts with the line

\references{dissertation}

